\documentclass[nobib]{tufte-handout}

\usepackage{amsmath}
\usepackage[utf8]{inputenc}
\usepackage{mathpazo}
\usepackage{booktabs}
\usepackage{microtype}
\usepackage{tikz}
\usepackage{enumitem}
\usepackage{pgfplots}
\pgfplotsset{width=9cm,compat=1.13}
\usepackage{pgfplotstable}

\title{Comparison of Three sum algorithms}
\author{Riko Jacob (rikj@itu.dk)}

\begin{document}
\maketitle

\section{Compared Algorithms and Impementations}

We compare the naive triple loop implementation and a dictionary/HashMap based implementation, in both python3 and java.
The algorithms are explained elsewhere.
 

\subsection{Experiments}

The two algorithms are
\begin{itemize}
\item enumerating all triples in $O(n^3)$
\item storing the numbers in a hashtable and check all pairs in $O(n^2)$ time
\end{itemize}

Both algorithms are implemented in java and python3.


\newcommand{\tableDir}{Tables56c1cb7}
We report on an experiment executed on a Thinkpad x260 with a i7-6500U CPU, with a  nominal frequency of 2.50GHz.
We run serial programs, even though the CPU has for cores.
The data can be found in \tableDir.
The running times are determined by a python script running the testprogram as a subprocess.
We report mean and standard deviation from 4 runs.
We stop executions that take more than 90 seconds.

The data for the triple loop algorithm in python looks like this (column 0 is size, 1 is mean, 2 is standard deviation):

\pgfplotstabletypeset{\tableDir /WeedJavaSimple.table}

\begin{tikzpicture}
	\begin{axis}[
		title={Running times no triples, data in \tableDir},
		xlabel={$N$},
                xmode = log,
                log ticks with fixed point,
                ymode = log,
		ylabel={Time (s)},
		xmin=30, xmax=15000,
		ymin=.03, ymax=75,
		xtick={30,50,100,200,400, 800, 2000,5000,15000},
		%ytick={0,40,80,160},
		ytick={.05,.1,.2,.5,1,2,5,10,20},
                legend style={at={(1.1,0)}, anchor=south west},
                % legend pos=north east, %north west,
		%ymajorgrids=true,
		%grid style=dashed,
	]
        %	coordinates { (100,32)(200,37.8)(400,44.6)(800,61.8)(1600,83.8)(3200,114) };
	\addplot[ color=red, mark=*,error bars/.cd,y dir=both,y explicit ]	table [x index=0, y index=1, y error index=2] {\tableDir /WeedPythSimple.table};
        % coordinates { (100,132)(200,72.8)(400,144.6)(800,161.8)(1600,133.8)(3200,224) };
	\addplot[color=blue, mark=x,error bars/.cd,y dir=both,y explicit ] table [x index=0, y index=1, y error index=2] {\tableDir /WeedJavaSimple.table};
	\addplot[ color=brown, mark=x,error bars/.cd,y dir=both,y explicit ] table [x index=0, y index=1, y error index=2] {\tableDir /WeedPythDict.table};
	\addplot[ color=green, mark=o,error bars/.cd,y dir=both,y explicit ] table [x index=0, y index=1, y error index=2] {\tableDir /WeedJavaDict.table};
%        \addplot[color=yellow] expression[domain=8:3200] {.00009*x+.08};
%        \addplot[color=yellow] expression[domain=8:3200] {.0009*x};
%        \addplot[color=yellow] expression[domain=8:3200] {.0002*x*ln(x)};
        \addplot[color=yellow] expression[domain=8:15000] {.000000031*x*x*x};
        \addplot[color=yellow] expression[domain=8:15000] {.00000000011*x*x*x};
%        \addplot[color=yellow] expression[domain=8:3200] {.00000000031*x*x*x*x};
        \addplot[color=yellow] expression[domain=8:15000] {.00000013*x*x};
        \addplot[color=yellow] expression[domain=8:15000] {.000000018*x*x};
	\legend{Python, Java, Fast Python, Fast Java, $x^2$, $x^3$}
	\end{axis}
\end{tikzpicture}


\end{document}
